\section{Political Reflection}
\label{s:poli_reflect}
%2-5 pages reflection on how the advice in main report would translate into a real-world setting.
\subsection{Introduction}

Decision-making concerning large scale infrastructure projects such as RfR benefit immensely from exploratory modelling approaches. It allows us to deepen insight of systems in question, explore behaviours under a wide range of uncertainties and scenarios, and using this understanding formulate robust and effective policies to address these uncertainties, which would otherwise be left to chance (Bankes, 1993). As analysts we are responsible for ensuring that the use of models and the information derived from them is correctly understood and utilised appropriately to aid in real life decision-making 
\parencite{pielke_honest_2007,van_enst_towards_2014}.

Through the process of analysing this case, in addition to the simulated debate-style policy negotiation, a number of tensions and challenges arose that might affect the successful adoption of proposed policies. Here, we reflect on three of these challenges, describe how our process of analysis attempts to address these, and additional actions that could still be taken. Finally, we reflect on the risks of adopting these strategies, and how our client, Gorssel, might adapt their approach in light of these risks.

\subsection{Tensions and Challenges} 

Three key tensions and challenges that were identified in the analysis process will be discussed in the following subsections. 

\subsubsection{Information Asymmetry (IA)}
Unequal levels of information and technological expertise of the different actors involved can mean that the chosen policy does not align with perceived understanding of modelling or goes against the interests of less informed actors. It represents an “ethical threat” \parencite{albertus_impact_2019}, that stems from inadequate information sharing between “information-rich and information-poor parties (vertical IA) or information that is distributed and incomplete among parties (horizontal IA) \parencite{clarkson_information_2007}. This phenomenon was visible during the debate preparation and negotiation process, where high-level actors such as the Rijkswaterstaat and transport company had teams of analysts to support them, resulting in vertical IA. These actors may not act in good faith or may hide relevant information from other actors creating more uncertainty for our modelling purposes. Moreover, they could use their expertise and reputation to legitimise their claims. We were provided with a mandate that was, in general, less technically focused than those of other actors. This made participation in policy debates and requests from other stakeholders for a quantification of our position a more challenging prospect, with more uncertainties to account for. 

\subsubsection{Intangibility of Costs and Benefits}
Models which are designed based on achieving consensus between actors may fail to adequately capture or quantify intangible/ephemeral costs benefits and objectives, which may influence behaviour of certain actors. Over the past decades the urban-rural divide has caused a divergence in the type of attitudes and political discourse between these factions. Rural discourse emphasises social, cultural, ecological, scenic and other ‘normative’ values of the countryside \parencite{frouws_contested_1998, andersson_beyond_2009}. This is reflected in Gorssel’s policy mandate which highlight their pride in organic and sustainable farming practices. More importantly, there was an expectation of ‘fairness’ in the treatment of Gorssel relative to urban municipalities (mainly Deventer), a factor that is not formally part of the model. Furthermore, model outputs cannot account for actions that actors may take in the interest of relationship management. Given they are both administered by Overijssel’s provincial government, in reality, Deventer and Gorssel may make concessions to one another in the interests of maintaining good working relationships for the long term. These concessions are not simple to quantify in a model. 

\subsubsection{Fixed Goals}
Goal definition is needed during decision-making processes to provide direction for any given project. Goals are the result of the first step of any decision-making process: problem scoping and formulation \parencite{enserink_policy_2010}. However, cognitive bias often leads actors to fixate on these predefined goals, even when through exploratory modelling analysis the problem understanding changes. This severely limits opportunities for the actor to engage with more expanded multi-issue agendas in the later stages of policy analysis that modellers present. This fixation may also be a strategic choice, for smaller actors to make a strong stance in order to obtain greater concessions from a high-level actor (e.g. the Rijkswaterstaat) than letting goals be dictated by analysts, who can have biases or agendas of their own \parencite{hans_de_bruijn_mark_de_bruijne_ernst_ten_heuvelhof_politics_2015}. 

\subsection{Effects on the Modelling Approach}

The three challenges were essential drivers to our modelling approach. Each challenge impacted our approach in the following ways: 

\subsubsection{Information Asymmetry} IA is inherent to any political decision-making problem. Yet starting actors out with the same model already provided a first mean to combat information asymmetry. For this exercise all actors/analysts were equipped with equivalent training in the use of modelling approaches. This ensured a predefined consensus on deep uncertainties and that all actors acknowledge that something must be done by cooperating. By employing a collectively accepted model even actors with less technological capabilities are able to engage with the analytical approach.

We tried our best to account for IA in our exploratory modelling approach and to conceive standpoints or problem formulations for actors, mainly within the Overijssel region, using actor analysis \parencite{enserink_policy_2010}. Specific to our modelling, the objectives of each actor were then scoped to only a small number of objectives per iteration (e.g. minimise deaths, damages, and costs) to not only reduce computational requirements, but it also ensure that the most critical trade-offs modelled are well communicated to a non-technical stakeholder/decision maker. 

\subsubsection{Intangibility of Costs and Benefits} Expanding the model to include additional model parameters to try and capture these “intangible” factors” was not seen as solution to address this challenge. This is because the usefulness of model is a balance between enough parameters to capture system behaviour, but also few enough to keep uncertainties manageable \parencite{saltelli_five_2020}. Moreover, a model does cannot and need not fully capture the real system, as some models may lead to alterations in system behaviour (e.g. as is the case for RfR models being used to alter riverflow), and will thus become self-invalidating.

We opted instead to simply introduce an information reporter to Gorssel’s problem formulation, which calculates the differences in expected annual damages and expected number of deaths between Gorssel and Deventer, as a proxy for the more intangible feature of ‘perception of fair treatment’. This reporter was then used in the final analysis of robustness for Gorssel, whereby the factors were used to inform both satisficing and regret-based robustness objectives \parencite{mcphail_robustness_2018}. In this way our advice would address important contextual factors for our client stakeholder that do not directly concern flood risk management. 

\subsubsection{Fixed Goals} In order to avoid rigid goals that reduces the potential policy space, the RDM approach was carried out. RDM is inherently iterative and should from the outset encourage actors to adjust their scope/objectives in light of unexpected outcomes through multiple iterations of the RDM process \parencite{lempert_general_2006}. 

For this assignment, following the policy debate, the problem formulations were updated and reframed, as a result of new information brought up by other actors. For instance, after learning that Deventer wished to impose a hard constraint to prevent the dike heightening in their city, we updated the problem formulation for that actor prevent policies that involve dike heightening in Deventer. 

\subsection{Solutions for Real World Situations }

For real world situations there exists a much wider range that can address these tensions and challenges that were not available to us during this project report. Below we present four promising ways/ideas that can guide modeller behaviour in future decision-making processes like RfR.

\subsubsection{Use of Serious Gaming to Build Consensus}

Using interactive methods such as serious gaming can help to further consolidate consensus in an environment where not all stakeholders have the same level of knowledge or technological capacity. It can also serve as a means to build empathy between actors, by asking them to take on the role of a different actor in the same problem domain. This could then lower the barrier to entry, such that a broader range of lower-power stakeholders can be truly involved in the decision-making process (for example farmers and/or citizens in the municipality of Gorssel) \parencite{savic_serious_2016}.

A serious gaming approach would require the development and use of a “blokkendoos” or “planning kit”, in essence a simplified model designed to aid various stakeholders rapidly assess spatial measures for inclusion within adaptive flood protection management strategies, without needed formal modelling experience \parencite{warren_collaborative_2015}. This approach would reduce expertise asymmetry, meaning there is less incentive for less technical actors to oppose model outputs or attack the model as whole. The empathy-building and social value of serious gaming approaches may also help to reveal intangible costs and benefits. 

\subsubsection{Support the Creation of a Multi-Issue Agenda} 

A fixed goal that opposes another actor’s goal will lead to serious conflicts within the political arena. Conflicts are inevitable, but the introduction of new ‘issues’ to the agenda widens negotiation space, allowing for opportunities to find trade-offs, agreements and compromises, as well as opportunities to broaden the scope beyond the model analysis to help prevent actors from becoming too attached to specific goals or problem formulations throughout the policy process \parencite{hans_de_bruijn_mark_de_bruijne_ernst_ten_heuvelhof_politics_2015} Creating a multi-issue agenda can also encourage cooperation when certain objective goals cause actors to behave in a non-cooperative manner, which in turn can also help reduce information asymmetry \parencite{coehoorn_learning_2004}.

\subsubsection{Reconceptualise Modelling with Dynamics Adaptive Policy Pathways}

With the acknowledgement of deep uncertainties, such as climate change, population growth, new technologies, economic developments, societal perspectives, preferences and stakeholders’ interests, modelling can no longer be viewed as a predictive planning tool. Rather, a new planning paradigm has emerged: dynamic adaptive policy pathways (DAPP), which involves designing alternative policy options that are dynamically adapted depending on circumstantial factors. Central to the approach is the monitoring of a meaningful metric or “signposts”, that can trigger the change from one policy option to another \parencite{haasnoot_dynamic_2013}. Since the MSMORDM approach results in a collection of promising policies, these may be combined into a DAPP framework, whereby the most appropriate policy is implemented at appropriate time steps \parencite{kwakkel_developing_2015}. This provide all actors to carefully adjust goals and analysts with more time to exchange information and re-evaluate if certain assumptions still hold. 

\subsubsection{Constructive Decision-Making \& Quantitative Storytelling}

Another important idea concerns not necessarily the modelling approach, but the general communicative approach itself. For the analysis of a complex system the modeller must take a secondary role of being an "honest broker". The "honest broker" one engages in a constructive style of decision-aiding, which involves the analyst facilitating joint sense-making with the client (in this case, Gorssel), and with relevant stakeholders \parencite{tsoukias_decision_2008}. Even the finest model is useless if its insights are not communicated and comprehensive in a way that impacts decision-maker behaviour. The human mind is not built to comprehend large sets of data, but more so images and stories. It is natural then that in general “numbers don’t stick, stories do”, meaning that more often than not, policy debates/decisions are determined by a leading narrative or anecdotes as opposed to empirical data \parencite{kettl_making_2016}.  

Given that a narrative is more effective in informing/guiding evidence-based policy debates than numerical data, policy analysts and modellers must find ways to work with this human limitation by tying data analysis to an overarching narrative. This is what is known as quantitative storytelling, which has manifested itself in the use of composite indicators, which are numerical metrics built to ‘tell a story’. For example, a simple composite actor is GDP and common narrative ties to progress and development \parencite{kuc-czarnecka_quantitative_2020}. Such metrics for flood risk can also be used to guide project implementation and address info asymmetry, since QS is more intuitive and inclusive for actors to follow. However, like GDP such composite indicators are criticised and come with considerable risks, discussed in the following paragraph \parencite{kuc-czarnecka_quantitative_2020}).

\subsection{Reflection on Proposed Strategies}

In the final section, we reflect on the potential risks which analysts may face and how the proposed solutions can be adapted, see Table \ref{tab:pr-risks-and-responses}.

\bigskip

\begin{table}[h!]
\caption{Risks of Strategies and their Responses}
\label{tab:pr-risks-and-responses}
\centering
\begin{tabular}{p{0.5\textwidth}|p{0.5\textwidth}}
\hline
Risk of Strategy &
  Potential Response \\ \hline
RDM strategies still requires a decision and investment to be made, which (in the context of wicked problems) will likely be irreversible &
  Revisit the modelling using a DAPP approach to find if there are opportunities for generating more flexible \\ \hline
Requirement to iterate over RDM strategies may never resolve to a favoured solution by all actors. &
  Ensure there is a ‘stopping point’ for the iteration over the problem scoping. This could be based on a deadline date at which a decision must be made, or a certain number of iterations. \\ \hline
Changes in leadership or government of involved actors results in changing priorities or disagreement with the consensus model. &
  Ensure that each step of the process is well documented, including all decisions and assumptions, as well as past iterations of the scoping and their outcomes, such that the new or changed actor can become up to date on the process. \\ \hline
Participating actors have ‘pet solutions’ and they may influence the serious gaming or constructive modelling approaches to favour their solution. &
  \citeauthor{pot_what_2018} propose a number of strategies to compensate for these behaviours, including the use of visions and scenarios. \\ \hline
Robust decisions can still be subject to other complications in their implementation as the model does not take into account limitations and uncertainties in implementation (for instance, delays in the actual building of flood mitigation measures) &
  Engage in expectation management with the clients and stakeholders around the limitations of the model. The model is just an abstraction of reality built to help inform decision making, scenarios are not forecasting. \cite{saltelli_five_2020} recommend that modellers be aware of hubris, ensuring they do not confer too much certainty in the modelling process. \\ \hline
The ‘policy window’ may not be open for the solutions to be implemented. \cite{rijke_room_2012} highlighted the importance of a sense of urgency which guided involvement in policy making. & Being aware of socio-political environment and long-term planning will allow rapid implementation when opportunities present themselves.
   \\ \hline
Even when all stakeholders use the same model, different problem formulations can result in significant differences in the nature of the policy advice. &  
  The Serious Gaming approach can enable stakeholders to see different how different conceptualisations can generate different results in the same model. \\ \hline
Quantitative storytelling using composite indicators may oversimplify multidimensional problems, resulting in simplistic or misleading policy messages if poorly constructed or misinterpreted & Acting as an "honest broker" ensures joint sense-making between client and analyst, so that potential misconceptions can be addressed.\\ \hline
  
\end{tabular}
\end{table}

Even when the analyst is aware of these risks and their potential responses, the very nature of the wicked, highly contested problem spaces in which this analysis is occurring is such that uncertainties in the process itself could derail any one of these strategies, while also rendering the potential responses redundant. As an analyst, we must be practice what we break; to be highly adaptable to the dynamic and uncertain political environment we are faced with. 

The key is to uphold the trust and integrity of the modelling process. Managing expectations and the ability for a solution to be robust to different futures is essential to being an honest and transparent participant in the policy-making process. However, modellers must strike a balance between building stakeholder trust in their model, and managing these expectations by using the many tools and methods at their disposal. Expectation management can backfire, if it results in scepticism towards the outputs of the model. However, keeping all this in mind and utilising the proposed potential responses, a favourable outcome for Gorssel and other clients in the future can be found.