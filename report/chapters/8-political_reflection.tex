\section{Political Reflection}
\label{s:poli_reflect}

- 
%2-5 pages reflection on how the advice in main report would translate into a real-world setting.
%Re-watch Lecture 6 part 2

\subsection{Tensions and Challenges}
%Reflect on some tensions and challenges that may adversely impact how the proposed advice is used in decision making. Keep in mind that to answer this you have to be clear about your role as analysts in the decision-making process, otherwise you cannot define what is ‘improper’ use of your analysis.

% There are three or more distinct challenges identified and described well. In describing them authors draw on the course literature and general academic literature. These challenges are very clearly specified to the proposed policy advice using relevant literature and good arguments.

The implementation of the proposed advice can be limited by a number of factors in reality. Here, we reflect on the process and outcomes experienced in the policy debate, identifying a few key challenges, how our existing policy advice can overcome them, and what more must be done. The key tensions and challenges identified were:
\begin{itemize}
    \item The implementation of upstream strategies rendering our policies ineffective \textcolor{red}{consider moving this to the discussion in main report}
    \item If actors above our client in the hierarchy are not well-prepared, it will not matter how well-prepared our client is, if they are not capable of adequately comprehending or advocating for strategies.
    \item Information asymmetry and different levels of technological capacity between actors may mean that actual implementation doesn't match modelling or perceived strategies
    \item The model cannot adequately capture/quantify intangible costs and benefits which will inherently influence the behaviour of different actors.
    \item What is shared in 'plenary' settings does not offer a complete picture of other deals which might take place behind closed doors
    \item Going into a policy negotiation process with fixed goals is not always the most productive approach, as it may close the door to other potential solutions or expanded options \textcolor{red}{I AM TOUGH ON REFERENCE DE BRUIJN AND TEN HEUVELHOF}
    \item Bottom up vs top down planning and implementation of policies. Current scoping of the policy solutions to the Overijssel region may conflict with Rijkswaterstaats top-down planning. 
\end{itemize}

\subsection{What has already been done}
% Based on these tensions and challenges, briefly outline what you have already accounted for in your analysis to ensure that the impact of these tensions and challenges is limited.

%Multiple specific analytical steps are clearly identified as politically salient and linked to identified challenges. A clear strategy based on the role of the analyst is outlined, supported by the literature, and followed throughout. The impact of these decisions in terms of analysis is explored and taken into account.

\subsection{What needs to be done}
%Based on these tensions and challenges, briefly outline what you would do and how you would behave to ensure that the impact of these tensions and challenges is limited.
% Specific strategies for how to behave in the decision-making process are linked to identified challenges. This is based on course literature, general academic literature, the context of the Ijssel river, and very specific to the proposed policy advice.
%Lecture 10 - Challenges with monitoring

\subsection{Reflection}
%Reflect on your strategy: Is what you have done and what you would do during the process sufficient to make sure your advice has the desired impact? How so? Are there other tensions and challenges that could also be problematic?
%3 or more risks addressed appropriately and in detail, including justifications (using literature) for why these are difficult to address in this context. Includes potential adaptations and exploration of the impact of those adaptations.


%%%%%%%%%%%%%%%%%%%% NOTE %%%%%%%%%%%%%%%%%%%%%%%%%
% Construction costs for either project are not reflective of the total costs, as compensations, relocations and other factors are not accounted for.

% Dutch law allows for expropriation under stringent requirements, is that a realistic option? (https://repository.tudelft.nl/islandora/object/uuid:088365ef-9d1c-43f5-925a-339215733f53/datastream/OBJ/download)

% The agenda was controled by Overijssel and was not shared early in advance, nor input was asked for, so we had wildly different expectations for it.