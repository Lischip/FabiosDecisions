\section{Political Reflection}
\label{s:poli_reflect}
%2-5 pages reflection on how the advice in main report would translate into a real-world setting.
%Re-watch Lecture 6 part 2

\subsection{Tensions and Challenges}
%Reflect on some tensions and challenges that may adversely impact how the proposed advice is used in decision making. Keep in mind that to answer this you have to be clear about your role as analysts in the decision-making process, otherwise you cannot define what is ‘improper’ use of your analysis.

% There are three or more distinct challenges identified and described well. In describing them authors draw on the course literature and general academic literature. These challenges are very clearly specified to the proposed policy advice using relevant literature and good arguments.

\subsection{What has already been done}
% Based on these tensions and challenges, briefly outline what you have already accounted for in your analysis to ensure that the impact of these tensions and challenges is limited.

%Multiple specific analytical steps are clearly identified as politically salient and linked to identified challenges. A clear strategy based on the role of the analyst is outlined, supported by the literature, and followed throughout. The impact of these decisions in terms of analysis is explored and taken into account.

\subsection{What needs to be done}
%Based on these tensions and challenges, briefly outline what you would do and how you would behave to ensure that the impact of these tensions and challenges is limited.
% Specific strategies for how to behave in the decision-making process are linked to identified challenges. This is based on course literature, general academic literature, the context of the Ijssel river, and very specific to the proposed policy advice.
%Lecture 10 - Challenges with monitoring

\subsection{Reflection}
%Reflect on your strategy: Is what you have done and what you would do during the process sufficient to make sure your advice has the desired impact? How so? Are there other tensions and challenges that could also be problematic?
%3 or more risks addressed appropriately and in detail, including justifications (using literature) for why these are difficult to address in this context. Includes potential adaptations and exploration of the impact of those adaptations.