\section{Political Reflection}
\label{s:poli_reflect}
%2-5 pages reflection on how the advice in main report would translate into a real-world setting.
%Re-watch Lecture 6 part 2
Decision-making concerning large scale infrastructure projects such as RfR benefit immensely from exploratory modelling approaches. It allows us to deepen insight of systems in question, explore behaviours under a wide range of uncertainties and scenarios, and using this understanding formulate robust and effective policies to address uncertainties \parencite{}

However, the reality is that we must still do the politicz

\subsection{Tensions and Challenges}
% Reflect on some tensions and challenges that may adversely impact how the proposed advice is used in decision making. Keep in mind that to answer this you have to be clear about your role as analysts in the decision-making process, otherwise you cannot define what is ‘improper’ use of your analysis.
% There are three or more distinct challenges identified and described well. In describing them authors draw on the course literature and general academic literature. These challenges are very clearly specified to the proposed policy advice using relevant literature and good arguments.

As an 'honest broker' engaging in a constructive style of decision-aiding, it is the responsibility of the analyst to engage in joint sense-making with the client (in this case, Gorssel), and with relevant stakeholders \parencite{tsoukias_decision_2008}. In order to effectively engage in the decision process, it is important to identify tensions and challenges that could impact on a constructive political process. 
By engaging in the 

The implementation of the proposed advice can be limited by a number of factors in reality. Here, we reflect on the process and outcomes experienced in the policy debate, identifying a few key challenges, how our existing policy advice can overcome them, and what more must be done. The key tensions and challenges identified were:

\subsubsection{1 - Information asymmetry}
can farmers understand MORDM?
\subsubsection{2 - Intangible Costs}
can you really quantify costs of beautiful old church?
\subsubsection{3 - Fixed goals}
going into policy debate with fixed goals, limits negotiation space and backs one into corner

\begin{itemize}
    \item Information asymmetry and different levels of technological capacity between actors may mean that actual implementation does not match modelling or perceived strategies.
    \item A model cannot adequately capture/quantify intangible costs and benefits which will inherently influence the behaviour of different actors.
    \item Going into a policy negotiation process with fixed goals is not always the most productive approach, as it may close the door to other potential solutions or expanded options \parencite{de_bruijn_management_2018}. While the current approach takes into account perspectives of other actors, it does require the analyst to specify goals/objectives of the client prior to the policy-building process.
     
    % David: The agenda was controlled by Overijssel and was not shared early in advance, nor input was asked for, so we had wildly different expectations for it.
\end{itemize}

\subsection{What has already been done}
% Based on these tensions and challenges, briefly outline what you have already accounted for in your analysis to ensure that the impact of these tensions and challenges is limited.
%Multiple specific analytical steps are clearly identified as politically salient and linked to identified challenges. A clear strategy based on the role of the analyst is outlined, supported by the literature, and followed throughout. The impact of these decisions in terms of analysis is explored and taken into account.

%Come up with frames that cast Gorssel as the victim and Deventer (or any other actor) as the villain - (L) this was proposed, but idk how well it fits to the tensions found. 
A strategy proposed to the client (Gorssel), based on the identified challenge of information asymmetry (IDK I had to base it on one of the tensions above) in with other parties, is using strong political frames in the discussions/debates. The client might have the problem of information asymmetry due to the size of the municipality which gives a more tight budget for analysing/modelling the problem. To counter this the municipality could use the political frame, 'Hero-Villain-Victim' with most emphasis on the villain-victim part, to strengthen their arguments and numbers (HANS DE BRUIJN-2016?). Two suggested implementations of this strategy:
\begin{itemize}
    \item \textbf{Hero:} The hero part of the frame can be fulfilled by Gorssel. In this strategy Gorssel can be the hero that defends their own 'poor' farmers that are being oppressed by the big city. Their work is again not valued as important as the city folk.. 
    \item \textbf{Villain:} The villain part of the frame can be fulfilled by Deventer. In this strategy Deventer is the villain as they value their city and their 'old' buildings higher than the hard working farmers beneath them. They expect others to make sacrifices that they do not want to make...
    \item \textbf{Victim:} The victim part of the frame can be fulfilled by Gorssel's Farmers. In this strategy the 'poor' farmers  are being oppressed by the big city. Their work is again not valued as important as the city folk..
\end{itemize}

Already taking into account perspectives of other actors.

\noindent A different approach might be to drop the third party (Hero) and just keep villain / victim with an additional option to keep the hero position as a possibility for the provincial government of Overijsel if they 'side' with Gorssel. The strategy is very similar to the variant described above, but allows for a wider play considering the provincial government. 
\begin{itemize}
    \item \textbf{Villain:} The villain part of the frame can be fulfilled by Deventer.
    \item \textbf{Victim:} The victim part of the frame can be fulfilled by Gorssel.
\end{itemize}

These strategies can be strengthened with the model results that Gorssel is able to achieve. 
\subsection{What needs to be done}
%Based on these tensions and challenges, briefly outline what you would do and how you would behave to ensure that the impact of these tensions and challenges is limited.
% Specific strategies for how to behave in the decision-making process are linked to identified challenges. This is based on course literature, general academic literature, the context of the IJssel river, and very specific to the proposed policy advice.
%Lecture 10 - Challenges with monitoring

% This is a draft, basically this will need to be reevaluated after analysis, and then rewritten anyways to make all the section consistent.
\begin{itemize}
    \item Not leave the agenda setting process to the province is a smart strategy. Taking initiative and proposing points can ensure that the actor discusses the points that are important to them. A negotiation needs to happen to define the important points, before the important points are negotiated.
    \item Have Gorssel's "demands" written out in case the initial talks fall through and Overijssel - just as they did after the debate - requests a written version. This can also mitigate the feeling of not feeling understood.  
    \item Expand the reach out to communicate with actors beyond the province, which can not only reveal the broader political landscape to the actor, but also provide different options to manoeuvre in case negotiations don't go as planned.
    \item Get the client involved in the modelling as well - to capture more intangible benefits and costs that the model cannot.
    \item having multiple stakeholders collaborating on the model results in more constructive learning opportunities
    % NO GO SCENARIOS FOR OTHER ACTORS? WE ONLY HAVE IF OUR SOLUTIONS ARE ACCEPTABLE TO THEM (unless we have the time to expand our analysis)
\end{itemize}

\subsection{Reflection}
%Reflect on your strategy: Is what you have done and what you would do during the process sufficient to make sure your advice has the desired impact? How so? Are there other tensions and challenges that could also be problematic?
%3 or more risks addressed appropriately and in detail, including justifications (using literature) for why these are difficult to address in this context. Includes potential adaptations and exploration of the impact of those adaptations.

Looking at the entire player actor field, it becomes clear that there is no clear-cut way to ensure that the given advice always works out. However well-prepared Gorssel is, the fact remains that Gorssel itself has little to none decision making power, and there are two main dependencies of Gorssel that affect the final outcome regardless of preparation: 
\begin{itemize}
    \item \textbf{The representation by Overijssel} is dependent on how well they fare in the plenary debate with all provincial and national actors. Even if that Gorssel may have perfectly conveyed their "demands" to Overijssel, Overijssel may fail to adequately do the same. 
    \item \textbf{The positioning of Gelderland} is something Gorssel has no direct influence upon. Because Overijssel is downriver from Gelderland, the possibility exists that they might strong arm Overijssel into following their preferred water management method. 
\end{itemize}

%Separate risk section or just leave it here?
There are also risks associated to the strategies proposed, for instance, broadening the outreach runs the risk of thinning resources too much, or producing an overload of information. Likewise, agenda setting is a responsibility of the province, and trying to influence that might prompt the province to think the actor is overstepping. This would be negative since the strategy taken is a very relationship-based one. This is also a risk that comes with the framing strategy - if this is done too frequently and without regard, it might lead to over-framing. Over-framing the problem may lead to Gorssel being taken less seriously in the future, meaning that the cost of over-framing isn't incurred immediately but in future interactions with the same actors.

\subsection{Conclusion}

% Let's just consider later if we do need this
Gorssel's position is defined not only by its objectives but also by its lack of real power in the decision making process. Gorssel can try to mitigate this by expanding on its influence, but at the end of the day it still remains powerless regarding the immediate outcome, and its quest to further influence the debate is not an easy nor risk free strategy.

%%%%%%%%%%%%%%%%%%%% NOTE %%%%%%%%%%%%%%%%%%%%%%%%%
% Construction costs for either project are not reflective of the total costs, as compensations, relocations and other factors are not accounted for.

% Dutch law allows for expropriation under stringent requirements, is that a realistic option? (https://repository.tudelft.nl/islandora/object/uuid:088365ef-9d1c-43f5-925a-339215733f53/datastream/OBJ/download)

% The agenda was controled by Overijssel and was not shared early in advance, nor input was asked for, so we had wildly different expectations for it.