\section*{Executive Summary}
\label{s:exec_summ}
%1-page summary with advice for problem owner; what advice do you give the problem owner and why? This advice should be understandable for a general audience unfamiliar with deep uncertainty methods and techniques.
% Convincing advice in light of multi actor context, understandable for non-experts
\begin{multicols}{2}

Located on the banks of the IJssel River - a river with a history of flooding - a plan for flood risk management is crucial to Gorssel. When coming up with a plan to mitigate these dangers, however, multiple stakeholders have to be taken into account. These stakeholders make up a complex actor arena with different objectives, which results in conflict and no one preferred outcome. Paired with the uncertainties and the decision making processes on local and provincial level, many uncertainties are involved when deciding on a strategy. Therefore, all decisions have to made under deep uncertainty. 

In this report, a policy strategy for Gorssel during the talks about flood protection of the province of Overijssel talks is proposed. The research questions answered is \vspace*{\fill} 
\begin{quote} 
\centering 
\textit{How can Gorssel (Dike Ring 4) affordably protect it’s citizens and businesses from flood risk to a similar level as neighbouring urban areas, while still preserving and protecting farmland from encroachment?}
\end{quote}
\vspace*{\fill}
To accurately assess the effectiveness of Gorssel's policy choices, the possible policy support for other Deventer and Overijssel will also be considered in order to increase feasibility. 

\bigskip

There are three ways flood risk can be minimised: Firstly, the dikes can be heightened, secondly the Early Warning System as a flood warning system and lastly, Room for the River can be utilised. When considering these options, Gorssel wants to minimise land encroachment to protect their farmers' livelihood while also ensuring "fairness" in the measures taken across the entire IJssel river. Furthermore, there are three key performance indicators (KPIs) for Gorssel to consider while looking for a suitable strategy. The KPIs consist of expected damage, expected deaths and the total cost of the implemented strategy over the IJssel river. To deal with multiple objectives and deep uncertainty, Exploratory Modelling Analysis (EMA) is used with the associated workbench. EMA enables the modeller to better understand the relation between uncertainties and policies, resulting in a robust outcome. To start off, uncertainties were identified followed by scenario discovery, where possible futures were found. With this, effectiveness of policies can be analysed and then tested for robustness, after which a recommendation can be done. 

\bigskip
\bigskip






% Results 
Robustness and uncertainty analysis from the perspective of Gorssel, Deventer and Overijssel revealed a selection of preferred policies for each actor:
\begin{itemize}
    \item Gorssel: Preferences policies with extensive investment in Room for the River projects and dike heightening in Deventer only, with no or modest dike heightening in Gorssel, and earlier warnings for evacuation.
    \item Deventer: Preferences policies with extensive investment in Room for the River projects in Gorssel, accompanied by either significant dike heightening in Gorssel or earlier warnings for evacuation.
    \item Overijssel: Supports low-moderate dike increases in both Gorssel and Deventer, with no investment in Room for the River projects, alongside earlier warnings for evacuations.
\end{itemize}


% Advice 
On the basis of these results, we recommend that the Municipality of Gorssel consider the following steps for flood risk mitigation:
\begin{enumerate}
    \item Modest dike increases in both Gorssel and Deventer in the near future present a robust solution to minimise flood risk in Gorssel.
    \item Gorssel should support the Overijssel regional government in lobbying for an increase of the Early Warning System to 3 days.
    \item The current scoping of the problem suggests that Room for the River projects are not favourable due to relative costs (compared to dike increases). However, the sensitivity of proposed policies to uncertainty in the probability of a dike failure suggests that these options may warrant further exploration as a policy option.
    \item We recommend that if Gorssel pursues dike heightening projects and EWS increases, that they complement these policies with other activities to improve their real-world robustness.
\end{enumerate}

\end{multicols}