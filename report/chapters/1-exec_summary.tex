\section*{Executive Summary}
\label{s:exec_summ}
%1-page summary with advice for problem owner; what advice do you give the problem owner and why? This advice should be understandable for a general audience unfamiliar with deep uncertainty methods and techniques.
% Convincing advice in light of multi actor context, understandable for non-experts

Located on the banks of the river IJssel - a river with a history of flooding - a plan for water management is crucial to Gorssel. When coming up with a plan to mitigate these dangers, however, multiple stakeholders have to be taken into account. These stakeholders make up a complex actor arena with different objectives, which results in conflict and no one preferred outcome. Paired with the uncertainties and the decision making processes on local and provincial level, many uncertainties are involved when deciding on a strategy. Therefore, all decisions have to made under deep uncertainty. 

In this report, a policy strategy for Gorssel during the talks about flood protection of the province of Overijssel talks is proposed. The research questions answered is \textit{How can Gorssel (Dike Ring 4) affordably protect it’s citizens and businesses from flood risk to a similar level as neighbouring urban areas, while still preserving and protecting farmland from encroachment?}. To accurately assess the effectiveness of Gorssel's policy choices, the possible policy support for other Deventer and Overijssel will also be considered in order to increase feasibility. 

\bigskip

\noindent There are three ways flood risk can be minimised: Firstly, the dikes can be heightened, secondly the Early Warning System as a flood warning system and lastly, Room for the River can be utilised. When considering these options, Gorssel wants to minimise land encroachment to protect their farmers' livelihood while also ensuring "fairness" in the measures taken across the entire IJssel river. Furthermore, there are three KPIs for Gorssel to consider while looking for a suitable strategy. The KPIs consist of expected damage, expected deaths and the total cost of the implemented strategy over the IJssel river. To deal with multiple objectives and deep uncertainty, Exploratory Modelling Analysis (EMA) is used with the associated workbench. EMA enables the modeller to better understand the relation between uncertainties and policies, resulting in a robust outcome. To start off, uncertainties were identified followed by scenario discovery, where possible futures were found. With this, effectiveness of policies can be analysed and then tested for robustness, after which a recommendation can be done. 

% Results 

On the basis of these results, advice on decision-making can be formulated for Gorssel. 
% Advice 