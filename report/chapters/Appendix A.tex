%Can be a link to a github repo
\section{Assumptions}
\begin{itemize}
    \item RfR costs include all relocation costs for farms and houses.
    \item Model-land objectives of Deventer and Overijssel are consistent with mandates revealed in the debate
    \item Gorssel is in the province of Overijssel
    \item The Rijkswaterstaat has the power and tools to conduct land expropriations under Dutch Law
    \item Used population figures from the last available census to make values for the difference function per capita rather than absolute figures. It was not considered to be important to find very precise information for this value.
    \item Room for the River Projects 3 and 4 were assumed to be in Gorssel and Deventer respectively - costs for the project were disaggregated accordingly.
    \item Population for the town of Lochem was used as a proxy for Gorssel as it was considered a more realistic population size.
    \item Z: do we assume Overijssel consists out of JUST Deventer and Gorssel/Lochem? L: I assumed the entire province, for the satisficing results, as otherwise we would have to make some weird decisions on the budget of Overijssel for water defence for two places.
    \item 10\% of the maximum total costs can be damages for an institution like, Deventer, Gorssel and Overijssel.
    \item 
    
\end{itemize}






% RfR also includes moving dikes, so doing RfR after dike heightening is silly
% We can include metromaps
% Deventer is DOWNSTREAM, Zutphen, Cortenoever and Doesburg are UPSTREAM
% Select objectives and pragmatically adjust based on possible policies and scenarios
    % Deaths - can be aggregated, you don't need to discount deaths
    % Damage - don't aggregate too soon - but if you do, you need to consider discount rates (and the implications of changing these - this is a critique for the discussion)
        %How correlated are EAD in different time steps? The more correlations there is, the more likely you can go to higher dimensions
% Look into algorithm options - Jazmin's work - BORG can go up to 17 different objectives
    % Also literature on scalarisation functions, where you repeatedly run an optimisation
% Is it humanly possible to optimise for 30 objectives at a time? No - so it doesn't make sense to try and model what you can't conceptualise.
    %Use archive logger to collect archive of epsilon progress, then separately, calculate the hypervolume from the archive.
    % MOEA - tend to fail when moving beyond highly stylised mathematical problems: Zatarain Salazar, J., P. M. Reed, J. D. Quinn, M. Giuliani and A. Castelletti (2017). "Balancing exploration, uncertainty and computational demands in many objective reservoir optimization." Advances in Water Resources 109: 196-210.
    % Zatarain Salazar, J., P. M. Reed, J. D. Herman, M. Giuliani and A. Castelletti (2016). "A diagnostic assessment of evolutionary algorithms for multi-objective surface water resevoir control." Advances in Water Resources 92(172-185).
    
%Explain what's in the table (problem formulations) and why.
Table \ref{t:objectives} summarises all assumed objectives, prior to the debate, from which the above three priority objectives were chosen. These were informed by stakeholder consultation ahead of the Room for the River debates, and through information on the mandate of Gorssel.

\begin{table}[h!]
\begin{tabular}{ll|lll}
\hline
         &                                   & Gorssel                 & Deventer                 & Overijssel \\ \hline
Gorssel  & Expected Annual Damage (Gorssel)  & -                       & \textgreater{}= Deventer & -          \\
         & Expected Annual Deaths (Gorssel)  & -                       & \textgreater{}= Deventer & -          \\
         & RfR Costs (Gorssel)               & -                       & Na                       & -          \\
         & Dike Costs (Gorssel)              & -                       & Na                       & -          \\
         & Total Costs (Gorssel)             & -                       & Na                       & -          \\
Deventer & Expected Annual Damage (Deventer) & \textgreater{}= Gorssel & -                        & -          \\
         & Expected Annual Deaths (Deventer) & \textgreater{}= Gorssel & -                        & -          \\
         & RfR Costs (Deventer)              & Na                      & -                        & -          \\
         & Dike Costs (Deventer)             & Na                      & -                        & -          \\
         & Total Costs (Deventer)            & Na                      & -                        & -          \\
Doesburg    & Expected Annual Damage (Doesburg)    & \textgreater{}= Gorssel & Na & \textgreater{}= Gorssel/Deventer \\
            & Expected Annual Deaths (Doesburg)    & \textgreater{}= Gorssel & Na & \textgreater{}= Gorssel/Deventer \\
Cortenoever & Expected Annual Damage (Cortenoever) & \textgreater{}= Gorssel & Na & \textgreater{}= Gorssel/Deventer \\
            & Expected Annual Deaths (Cortenoever) & \textgreater{}= Gorssel & Na & \textgreater{}= Gorssel/Deventer
\end{tabular}
\caption{-}
\label{t:objectives}
\end{table}