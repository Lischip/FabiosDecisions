\section{Discussion}
\label{s:discussion}
%Discussion; 

%what are key threats to validity of your conclusions? 
%Robustness metrics? McPhail paper

%What directions do you see for further refining or improving your analysis? Again, this should be grounded in literature and/or an awareness of the decision arena.

% Identifies key limitations of their analysis from both a methodological point of view and the policy arena point of view. Discusses their implications for conclusions and suggests ideas for future work to overcome them

\subsection{Methodological Limitations}
In the analysis process, we made several methodological decisions to enable the effective modelling of the problem posed by Gorssel. While the key decisions have been justified in Section\ref{s:approach}, a few methodological threats to the validity of this report's policy advice remain.

\subsubsection{Level of Aggregation in the Model}
In choosing the problem formulations for Gorssel, Deventer, and Overijssel, a number of key variables in the model were either aggregated or disaggregated, to achieve a problem formulation that adequately articulated the actor's objectives. Indeed, there is a trade-off between higher levels of aggregation, which may allow for more information to be embedded within fewer objectives, and lower levels of aggregation, which introduces more bias into the problem formulation. \citeauthor{kasprzyk_battling_2016} (\citeyear{kasprzyk_battling_2016}) explain that aggregated, lower objective approaches to modelling might bias decision support tools. Due to limitations in computation and explainability for clients, some of the objectives had to be aggregated in our modelling approach (for example, Overijssel's problem formulation had an objective of combined deaths for Deventer and Gorssel, rather than optimising them separately).

The aggregation of outcomes over all planning steps also limits the type of policy advice we were able to provide. By aggregating these outcomes, we are unable to consider the timing of decisions in our policy analysis, limiting our ability to consider reversibility and timeliness of decisions necessary for presenting advice in the form of Dynamic Adaptive Policy Pathways \parencite{kwakkel_coping_2016,marchau_decision_2019}.

It is also feasible that this aggregation created bias related to the influence of different uncertainties. Uncertainty analysis revealed that many of the uncertainties had little to no influence on the outcomes for the actors considered, with Probability of Failure of the Gorssel dike being the most consequential factor for the outcomes in the Gorssel analysis.

\textcolor{red}{Conceptual aggregation - Balance between complexity and simplicity in the model - e.g. climate change isn't included in the model, international actors not included, system scoping - changing the scoping will allow for exploration of whether these conclusions remain valid}

\subsubsection{Validation with stakeholders}
The assumptions applied in the development of problem formulations were not validated with input from the client (Gorssel) or from the 'rival' stakeholders. This meant that it wasn't possible to sanity-check the proposed policies, or revise the problem formulations in light of expert input. As a result, only one iteration was performed over the complete Multi-Scenario MORDM process to generate the policy advice presented in Section \ref{s:results}. This may limit the relevancy and robustness of the proposed policies, as well as likelihood of acceptance by stakeholders \parencite{quinn_rival_2017}.

Further to this, some assumptions were made about the objectives and constraints for stakeholders which influenced the way in which the problem formulations were structured. For example, in it's problem formulation, Deventer has a hard constraint on dike increases in it's policy formulation. This generated some unusual policies, which warrants further investigation

\subsubsection{Limited use of uncertainty analysis and scenario discovery}
The results of uncertainty analysis and scenario discovery from the preliminary round of MORDM were not employed effectively in the subsequent selection of scenarios. Time constraints in the analysis process meant that information on uncertainties was not adequately incorporated to adjust the uncertainty ranges over which selected scenarios were sampled. The results from these analyses could have been better incorporated into decreasing the dimensionality and uncertainty of the selected scenarios, which in turn, could have reduced computational intensity of other downstream tasks in the multi-scenario MORDM. This limitation may also offer an explanation for why the final proposed policies remain highly sensitive to a single input factor (probability of dike failure at Gorssel).

\subsubsection{Limitations of RDM}
RDM is a strong approach for handling wicked problems, but is not without it's own limitations \parencite{kwakkel_coping_2016}. RDM enabled us to investigate the trade-offs of different policies, but this method is limited in it's adaptability once implemented. Further to this, Rijkswaterstaat advocates adaptive planning (\textbf{SOURCE??}), and policy proposals, even at the municipal level, should look to come in line with national approaches.

\subsubsection{Creation of the 'Difference' objectives}
In translating Gorssel's mandate from problem framing into model problem formulation, we originally looked at minimising on the difference between damage in Gorssel and Deventer. This was proposed as a means of quantifying the 'fairness' of different policies in terms of effects in rural and urban areas. However, after further consideration, we decided that minimising this as an objective was arguably too Machiavellian, given that minimising this to below zero is effectively choosing scenarios that maximise deaths in Deventer, often without any additional benefit to Gorssel.

This objective was instead retained for calculation of satisficing robustness metrics, to capture the intangible concepts of fairness and equality between urban and rural areas. However, this decision meant that the objective is more for the benefit of the client and analysts, and does not add any analytical rigour to the definition of 'fairness'. Further to this, the difference value in optimisations is always in favour of Gorssel, but it remains uncertain that this covers the real inequity between the rural and urban actors. For example Gorssel probably receives a lot less revenue, so damages there are probably more impactful than in Deventer.


\subsection{Policy Arena Limitations}

The policy arena was demarcated in section \ref{s:prob_frame}, however the manner thereof may leave some blind spots and create limitations which have to be addressed. These concerns will be addressed in the following subsections.  

\subsubsection{Scoping of the model versus real life}
To reiterate, the manner of scoping is that we have merely looked at decision making within the province of Overijssel. This means that the implementations of upstream dike rings are not considered. However, in reality, the implementation of upstream strategies have the power to render our found policies ineffective (EXAMPLE?). Besides this, the decision made within our defined scope is influenced by and has an influence on the choices other actors make, especially in a financial manner. This has to do with the "fairness" we seek for Gorssel, for others also seek this same "fairness" for their own dike ring. For instance, Dike Rings 1 to 3, located upstream, care about the funds being made available for projects for Dike Ring 4 and 5, as this means there will be less funds available for the projects upstream. Finally, the scoping means we do not see the complete consequences of the Early Warning System. The EWS is a "global" policy - meaning that it would affect all actors and not merely the one within our scope. In that sense, the decision made here to use it would mean that we should also convince the other actors instead of just the ones within the scope as they are also affected by it. 

% scoping vs real life:
% Overijssel is not only gorssel and deventer, only abstraction
% future model iterations should include more
% so that framing comes closer to reality for province

\subsubsection{Decision Making Processes}
The current scoping of the policy solutions may conflict with Rijkswaterstaat's actual manner of top-down planning. Conflicts between bottom-up and top-down planning and implementation of policies may create inefficiencies, or results in missed opportunities (CITE: Koontz). For Gorssel, the focus of objectives was concerned with preparing a strategy on how to develop a proposal with the province of Overijssel. This means that while the analysis is relevant in this context, it may fail to account the reality of how Rijkswaterstaat functions. Similarly, the decisions made in the model cannot always be implemented in the order suggested. (continue this if actual policy suggests this) For instance, when dike strengthening is done first and Room for the River follows after, the dikes would have to be dismantled. 

\subsubsection{Dependency on Models}
To better serve Gorssel, the base model given by Rijkswaterstaat was adjusted, such as the disaggregation of costs of Room for the River projects. This was done to better suit Gorssel, but poses a problem when other actors in the policy arena use the same (base) model or have also made small adjustments, as we may not be able to assume a consensus exists among all actors, as \cite{kwakkel_coping_2016} writes is important to have. All findings should be treated with a certain wariness so they're not used to support actors' agendas. Models are frequently misused to support agendas \parencite{saltelli_five_2020}. A way to prevent this negotiated nonsense is validation through stakeholders and experts. 

\subsection{Proposed Improvements and Further Work}
Proposed solutions for each of these limitations are considered in light of current knowledge and available literature. These improvements are summarised in Table \ref{tab:Proposed Solutions}

\begin{table}[h]
\caption{Proposed solutions for each of the identified limitations}
\label{tab:Proposed Solutions}
\centering
\begin{tabular}{p{0.2\textwidth}|p{0.7\textwidth}}
\textbf{Limitation} & \textbf{Proposed Solution}  \\ \hline
Levels of aggregation & Future work should look to disaggregate factors (especially cost objectives) to identify whether the aggregation of these factors introduced bias in the policy analysis in favour of one approach or another. \\ \hline
Limitations of RDM and reversibility of decisions & Re-evaluate the problem for multiple time steps and apply DAPP. The selection of the style of proposed policy should ultimately match the client's risk tolerance and preferences regarding adaptive planning \parencite{marchau_decision_2019}. \\ \hline
Lack of validation with stakeholders &Additional iterations on the model specification and framing should be conducted in light of identified policies and scenarios. Having identified opportunities for coalitions (or tensions between policies of different actors), should iterate through the Multi-scenario MORDM process again following consultation with the client and stakeholders, re-specifying the problem formulations, or adding in new issues or objectives. For instance, running new problem formulations for actors in Gelderland province to test the policies against their objectives and preferences. \\ \hline
Limited application of uncertainty analysis and scenario discovery & Further iterations on the multi-scenario MORDM should better incorporate insights from the uncertainty analysis, such that a more specific set of scenarios are selected for the analysis and optimisation of the policies \parencite{watson_incorporating_2017, eker_including_2018}. \\ \hline
Creation of difference objectives & Investigate alternative robustness metrics through consultation with stakeholders and client. There are a range of metrics that could serve to underline robustness in terms of inequities in proposed policies \parencite{mcphail_robustness_2018}. \\ \hline
Scoping of the model relative to reality & Additional iterations at various scopes of the problem when policies have been selected may work well to show how the chosen policy would function under different scopes.        \\ \hline
Decision-making processes & A way to counter this would be to perhaps find a manner to be more involved in the final decision making process at provincial level   \\ \hline
Dependency on models &  Better communication between analysts and modellers would allow for aligning of the different models and perhaps allow for understanding of differences. Furthermore, transparency and naming the vulnerabilities of the models used can build trust. \parencite{saltelli_five_2020}               
\end{tabular}
\end{table}