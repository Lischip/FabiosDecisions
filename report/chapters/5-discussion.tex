\section{Discussion}
\label{s:discussion}
%Discussion; 

%what are key threats to validity of your conclusions? 
%Robustness metrics? McPhail paper

%What directions do you see for further refining or improving your analysis? Again, this should be grounded in literature and/or an awareness of the decision arena.

% Identifies key limitations of their analysis from both a methodological point of view and the policy arena point of view. Discusses their implications for conclusions and suggests ideas for future work to overcome them

Rdm enabled us to investigate the trade-offs of different policies, but this method is not adaptable. Rijkswaterstaat advocates adaptive planning. It would have been good to complement our approach with Dynamic Adaptive Policy Pathways as explained by \cite{kwakkel_coping_2016}.

Discussion stuff to remember:
\begin{itemize}
    \item EWS is a global policy across the whole region - this is dom
    \item The implementation of upstream strategies rendering our policies ineffective \textcolor{red}{consider moving this to the discussion in main report}
    \item Bottom up vs top down planning and implementation of policies. Current scoping of the policy solutions to the Overijssel region may conflict with Rijkswaterstaat's top-down planning.
\end{itemize}