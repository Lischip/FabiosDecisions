\section{Discussion}
\label{s:discussion}
%Discussion; 

%what are key threats to validity of your conclusions? 
%Robustness metrics? McPhail paper

%What directions do you see for further refining or improving your analysis? Again, this should be grounded in literature and/or an awareness of the decision arena.

% Identifies key limitations of their analysis from both a methodological point of view and the policy arena point of view. Discusses their implications for conclusions and suggests ideas for future work to overcome them

Rdm enabled us to investigate the trade-offs of different policies, but this method is not adaptable. Rijkswaterstaat advocates adaptive planning. It would have been good to complement our approach with Dynamic Adaptive Policy Pathways as explained by \cite{kwakkel_coping_2016}.

Discussion stuff to remember:
\begin{itemize}
    \item EWS is a global policy across the whole region - this is dom
    \item The implementation of upstream strategies rendering our policies ineffective \textcolor{red}{consider moving this to the discussion in main report}
    \item Bottom up vs top down planning and implementation of policies. Current scoping of the policy solutions to the Overijssel region may conflict with Rijkswaterstaat's top-down planning. Working with the client, the focus of objectives was concerned with preparing for workshopping a proposal at the level of Overijssel. This means that while the analysis is relevant in this context, it may fail to account for top-down decisions from Rijkswaterstaat.
    \item Translating problem formulation into problem framing - originally looked at optimising for the difference between damage in Gorssel and Deventer, but minimising this as an objective is arguably too Machiavellian (minimising this to below zero is effectively choosing scenarios that maximise deaths in Deventer).
    \item We are considering evacuation costs for Deventer so that we have a tradeoff that justifies MORDM. That's the practical explanation, but how do we put that in the report? No upper bound of zero on this because it is used only as additional information to colour the policy conversation
    \item The difference between Gorssel and Deventer is not used to optimise, but to capture the intangible concepts of fairness and equality between urban and rural areas. It's a concept to add not to the model but to the analysis. That difference seems to be always in favour of Gorssel though, but does that cover the real impact? For example Gorssel probably receives a lot less revenue, so damages there are probably more impactful than in Deventer.
    \item Costs are global, they are minimised by Gorssel and Deventer because even if no projects are done in their land, they can still be required to share the costs of projects upstream.
    \item MORO might be considered for further development of scenarios if more time or computing power is available.
    \item Additional iterations on the model specification and framing should be conducted in light of identified policies and scenarios. Having identified opportunities for coalitions (or tensions between policies of different actors), should iterate through the Multi-scenario MORDM process again, re-specifying the problem formulations, or adding in new issues or objectives. For instance, running new problem formulations for actors in Gelderland province to test the policies against their objectives and preferences.
    \item (Cite: Kasprzyk 2016) explains that aggregated, lower objective approaches to modelling might bias decision support tools. Due to limitations in computation and explainability for clients, some of the objectives had to be aggregated in our modelling approach. Future work should look to disaggregate factors (especially cost objectives) to identify whether the aggregation of these factors introduced bias in the policy analysis in favour of one approach or another.
    \item Beware negotiated nonsense - these solutions will need to be validated through consultation with stakeholders and subject matter experts.
    \item Actors can still care about the costs of things they are not paying because that means less money available to them anyways (i.e. an upstream city cares about the money spent in a downstream project because that means there is less money for upstream projects).
    \item The model does not necessarily have to reflect reality too closely, a good model is self invalidating because it changes the system.
    
    
    
    \item So when u perform dike-increase at planning step 1, u're not going to Rfr at planning step 2, u can only increase dikes up to a certain level (30 cm), u can't just keep increasing. Every increase in dike height also results in an increase in land encroachment.
    \item Also, whenever you want to switch to RfR programs, you have to take away part of the dike that you built, so strengthening dikes can be robust but not necessarily resilient and end up increasing costs.
\end{itemize}