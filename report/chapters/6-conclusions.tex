\section{Conclusions}
\label{s:conc}
%Conclusions; a more extended conclusion grounded in your results and discussion, leading to a clear advice for your problem owner.

% Convincing conclusions consistent with analysis advice is appropriate for problem owner and the multi-actor context

In this report we have applied state of the art decision-making under deep uncertainty approaches to identify robust and cost effective policy alternatives for flood risk management in the Municipality of Gorssel. This was done to answer the following question:
\begin{quote}
    How can Gorssel (Dike Ring 4) affordably protect it's citizens and businesses from flood risk to a similar level as neighbouring urban areas, while still preserving and protecting farmland from encroachment?
\end{quote} 

The Key Performance Indicators (set of objectives) chosen for Gorssel consisted of the expected annual damage, the expected number of deaths and the total costs of the entire IJssel river area. Multi-scenario MORDM was employed to generate a large range of scenarios, and appropriate policies within these scenarios from the perspective of Gorssel, Deventer, and Overijssel. The identified policies were tested for their relative robustness, and compared to identify opportunities for potential coalition forming. Given the conflicts between the preferred policies of Deventer and Gorssel, Overijssel's most robust policies present options for compromise between the objectives of the three actors.

\subsection{Final Policy Recommendation for Gorssel}
% EWS for 3 days is likely to get consensus agreement across all three actors. One recommendation is to centre the discussion around this commonality:
Based on results synthesis it was found that an early warning system is likely to get consensus from all three actors within the Overijssel province. Therefore the first recommendation is to commence negotiations on this point to establish commonalities and a basis for upcoming cooperation.

\subsection{Next Steps}
Based on these results, and the points of discussion raised in Section \ref{s:discussion}, we strongly recommend re-visiting the problem formulations following policy discussions/negotiations with both Deventer and Overijssel. The results of the analysis for the three actors can be used to inform policy discussions, and a reformulation of a collective problem formulation for the three actors which can adequately represent the trade-offs inherent in the region. Given general consensus between Overijssel's actors regarding a three-day early-warning, we recommend that the next problem formulation fixes the Early Warning System lever at this value to search for more granularity in policy options among other levers.

\subsubsection{Pursuing Suggested Policies}
Following this re-evaluation, should Gorssel still wish to pursue local dike expansion options and increases to the early warning system, there are three key recommendations:
\begin{enumerate}
    \item Gorssel (or the Province of Overijssel) should seek additional expert advice/testing to reduce uncertainty regarding the probability of failure of dikes in both Gorssel and Deventer.
    \item Any dike expansions should be accompanied by a dike strength monitoring and evaluation program to manage risks of failure, and to ensure preventative actions can be taken to further fortify or repair identified weaknesses.
    \item Early Warning System increases should be accompanied by community engagement to ensure compliance with evacuation orders, and improved understanding of why evacuation orders may become more frequent.
\end{enumerate}

\noindent On a closing note, Gorssel should also keep in mind that all these results are meant to be utilised as support for the decision making process and not be seen as \textit{the} decision making process. Readers should acknowledge the methodological and political limitations of the approach taken in this analysis. With this manner of modelling, a wide range of potential futures could be explored and many uncertainties have been taken into account, resulting in possibilities but not predictions of the future.