\section{Results}
\label{s:results}
%Results; this should be a readable summary of the results from applying the approach. Don’t pursue death by figures but carefully select what visualisations (figures, tables) are functional for telling your story and logically lead to the main conclusions and policy advice?

% Convincing story, consistent with approach using carefully designed visuals and tables to support narrative
This section summarises the key results from applying this approach, first identifying the most robust policies for Gorssel, Deventer and Overijssel, and subsequently discussing options for synthesising policy choices between the three actors. 

\subsection{Robust Decision Making}
%Look at the policies for the five different scenarios and examine trade-offs using different robustness metrics.
We used two metrics for the robust decision-making process: Satisficing (the domain-criterion) and maximum regret. 
%#################################################################
%   REGRET AND SATISFICING - GORSSEL
%#################################################################
\subsubsection{Gorssel}
The results for Gorssel's satisficing analysis with the domain criterion are shown in \autoref{fig:domain_criterion_gorssel}. We analysed the twelve policies selected during policy subset selection for trade-offs and the extent to which the results satisfy Gorssel's threshold values. \newline
\autoref{fig:domain_criterion_gorssel} shows that every policy is satisficing for "Gorssel Expected Number of Deaths". This means that all potential policy recommendations ensure that the legal standards for citizen safety are met in all tested scenarios. 
We can also observe that Gorssel's G\_1 policy performs best for the domain-criterion results, reaching the highest domain-criterion scores on average. However, this policy comes with a significant trade-off with the expected annual damage as the policy often goes over the annual damage budget and thus reaches a low domain-criterion value for 'expected annual damage'. This would make the policy less robust in terms of damage and more in terms of costs. For the other policies, it is the other way around. Most policies except for G\_1 score 0 on 'total costs', and these policies are thus not robust when considering these satisficing results.  \newline

\noindent In \autoref{fig:regret_gorssel}, the results for Gorssel's regret analysis are shown. The least regret for Gorssel appears with policy G\_2. This is because the expected number of Deaths and the Damage for Gorssel have 0 regret in this policy; this comes at a high total cost. This is a clear example of the trade-offs that we must make within the policies. A trade-off like this is also shown with policy G\_8, where there is little regret for the damage, but the number of deaths and the costs leads to high regret.

Another noteworthy result is that some of the high regret policies have a low satisficing score and vice-versa.

\begin{figure}[H]
  \centering
  \begin{minipage}[b]{0.4\textwidth}
    \includegraphics[width=1.15\textwidth]{report/figures/results/domain_criterion_Gorssel.png}
    \caption{Results for Gorssel's domain criterion.}
    \label{fig:domain_criterion_gorssel}
  \end{minipage}
  \hfill
  \begin{minipage}[b]{0.4\textwidth}
    \includegraphics[width=1.15\textwidth]{report/figures/results/regret_figure_Gorssel.png}
    \caption{Results for Gorssel's maximum regret.}
    \label{fig:regret_gorssel}
  \end{minipage}
\end{figure}




%#################################################################
%   REGRET AND SATISFICING - DEVENTER
%#################################################################
\subsubsection{Deventer}
The results for Deventer's satisficing analysis with the domain criterion are shown in  \autoref{fig:domain_criterion_Deventers}. 
\autoref{fig:domain_criterion_Deventers} shows that every policy is satisficing for "Deventer Expected Number of Deaths" and for "Deventer Total Costs". This means that later policy recommendations, just like in Gorssel's case, are ensured that all policies will meet the legal standards for citizen safety in all scenarios and that the budget of Deventer never overextends (within the set thresholds). 

The policy D\_9, together with D\_8, performed the best with the satisficing scores for the expected damage. The other policies score only a little over half of this value. As such, policy D\_8 and D\_9 are the more robust policies. \newline 

\noindent \autoref{fig:regret_Deventers} shows the regret results for Deventer's policy options. Here it can be seen that policy D\_8 and D\_9 score the best on the 'average' regret with certain trade-offs. Especially policy D\_9 shows a big trade-off compared to D\_8 between expected damage and expected deaths versus the total costs that this brings with it. Another policy that reveals potential policy strategies for Deventer is policy D\_4. This policy shows that the only way Deventer will not 'regret' their total costs is when they trade costs for high damages and a high expected number of deaths. 
\begin{figure}[H]
  \centering
  \begin{minipage}[b]{0.4\textwidth}
    \includegraphics[width=1.2\textwidth]{report/figures/results/domain_criterion_Deventer.png}
    \caption{Results for Deventer's domain criterion.}
    \label{fig:domain_criterion_Deventers}
  \end{minipage}
  \hfill
  \begin{minipage}[b]{0.4\textwidth}
    \includegraphics[width=1.2\textwidth]{report/figures/results/regret_figure_Deventer.png}
    \caption{Results for Deventer's maximum regret.}
    \label{fig:regret_Deventers}
  \end{minipage}
\end{figure}

%#################################################################
%   REGRET AND SATISFICING - OVERIJSSEL
%#################################################################
\subsubsection{Overijssel}
Overijssel's satisficing analysis with the domain criterion is shown in \autoref{fig:domain_criterion_Overijssels}. Here, we can see that every policy is satisficing for "Gorssel and Deventer Expected Number of Deaths" and none except for O\_6 for "Gorssel and Deventer Total Costs". And thus, all potential policy recommendations can be sure that the legal standards for citizen safety will be met in all scenarios while crossing the total cost threshold of Overijssel, except for O\_6 because it does not satisfice for costs. 
Policy O\_8 and policy O\_10 are better choices, scoring low on regret based on their satisficing scores due to their scores on 'expected damage'. \newline

\noindent \autoref{fig:domain_criterion_Overijssels} shows the regret results for Overijssel's policy options. Here policy O\_8 and O\_10 score the best on the 'average' regret but show certain trade-offs. It is interesting to note that policy D\_10 shows almost no trade-off compared to the other policies.

\begin{figure}[H]
  \centering
  \begin{minipage}[b]{0.4\textwidth}
    \includegraphics[width=1.2\textwidth]{report/figures/results/domain_criterion_Overijssel.png}
    \caption{Results for Overijssel's domain criterion.}
    \label{fig:domain_criterion_Overijssels}
  \end{minipage}
  \hfill
  \begin{minipage}[b]{0.4\textwidth}
    \includegraphics[width=1.2\textwidth]{report/figures/results/regret_figure_Overijssel.png}
    \caption{Results for Overijssel's maximum regret.}
    \label{fig:regret_Overijssels}
  \end{minipage}
\end{figure}


\subsection{Sensitivity Analysis}

We used the policies that were deemed best from the robustness analysis to initialise the model to perform the sensitivity analysis as explained in \autoref{ss:sensitivity-analysis}. This allowed us to assess what uncertainties and levers significantly impact the outcomes when policies are in place. The analysis of these can be viewed per actor in the following section. The generated figures for the sensitivity analysis can be found in \autoref{a:sensitivity-analysis}.

%Firstly, we did the sensitivity analysis with feature scoring without the policies to grasp which uncertainty and levers the actors would be most sensitive to. After, policies are added and analysed to assess whether they would be sensitive or not. 

\subsubsection{Gorssel}
From the earlier sensitivity analysis and scenario discovery, as explained in subsection~\ref{ss:sensitivity-analysis-scenario-discovery}, it became apparent that Gorssel's outcomes are not only sensitive to the durability of their own dike but also to the durability of Deventer's dike. When adding the policies, we observe that the sensitivity of the outcomes to the durability of Gorssel's and Deventer's dikes remains the highest. 

\subsubsection{Deventer}

During the sensitivity analysis and scenario discovery step, we found that the biggest threat to damages and deaths for Deventer was their own dike durability. This was in line with expectations because only their own dike breaking would impact Deventer. And just as with Gorssel, we observed that the outcome was still least robust under the dike durability of their own dike when adding the policies. 

\subsubsection{Overijssel}
For the costs of the province of Overijssel, we observed that they are sensitive to dike increase. Costs were most sensitive to dike increase in Gorssel, which is in line with reality as this is the most costly procedure in the province, other than Room for the River. However, because none of the policies for Overijssel recommended Room for the River, there is no observed sensitivity. Furthermore, the entire province showed a higher score on deaths and damages across the province as well when it comes to dike failure. 

\subsection{Qualitative assessment of shortlisted Policies per Actor}
We identified a subset of the top five most robust policies (prioritising regret-based metrics). Here we present, a summary of each policy regarding dike heightening, Room for the River projects, and early warning systems for each actor. The damage, death, and cost outcomes are not included in these tables, as we have already optimised these policies to be within acceptable levels based on the robustness metrics in previous steps. We qualitatively interpreted for opportunities for coalition forming, with relevant policies highlighted yellow in each table.

Interpreting these policies for opportunities for coalition-forming, there is a commonly re-occurring objective of a three-day early warning, which presents an opening for consensus-building between actors in the Overijssel region. The tables below explore this in more detail.

\subsubsection{Gorssel}
The five most robust policies for Gorssel are presented in Table \ref{tab:gpols}.

\begin{table}[h!]
  \centering
  \captionsetup{justification=centering,margin=2cm}
  \caption{Robust policies for Gorssel. RfR stands for Room for the River, dike increases are in decimetres and aggregated over all planning steps, EWS refers to Early Warning System in days.}
  \label{tab:gpols}
  \includegraphics[width=0.8\linewidth]{report/figures/gpols.png}
\end{table}

\noindent Given that we optimised these policies for Gorssel's objectives, all of these policies are favourable for Gorssel. However, we highlight policies G1 and G2 for opportunities they present for coalition forming with Deventer and Overijssel. Arguably, these two policies are less favourable for Deventer, given the more significant investments in Room for the River and dike increases. However, they are well-aligned with many of Overijssel's preferred policies (seen in Table \ref{tab:opols}) which prioritise earlier warnings and modest dike heightening (although they do not include Room for the River policies).

\subsubsection{Deventer}
The five most robust policies for Deventer are presented in Table \ref{tab:dpols}.

\begin{table}[h!]
  \centering
  \captionsetup{justification=centering,margin=2cm}
  \caption{Robust policies for Deventer. RfR stands for Room for the River, dike increases are in decimetres and aggregated over all planning steps, EWS refers to Early Warning System in days.}
  \label{tab:dpols}
  \includegraphics[width=0.8\linewidth]{report/figures/dpols.png}
\end{table}

Deventer's policies offer the least to Gorssel in terms of potential coalition-forming/points for negotiation. Of interest, however, is policy D9. While requiring significant Room for the River investment in Gorssel, this policy does not require any dike heightening (as for all other robust policies for Deventer), relying instead on earlier warnings, similar to related robust policies of Deventer, Gorssel, and Overijssel.

\subsubsection{Overijssel}
The five most robust policies for Overijssel are presented in Table \ref{tab:opols}.

\begin{table}[h!]
  \centering
  \captionsetup{justification=centering,margin=2cm}
  \caption{Robust policies for Overijssel. RfR stands for Room for the River, dike increases are in decimetres and aggregated over all planning steps, EWS refers to Early Warning System.}
  \label{tab:opols}
  \includegraphics[width=0.8\linewidth]{report/figures/opols.png}
\end{table}

\noindent Overijssel's policies prioritise dike increases in the first planning step, with none of the policies prioritising Room for the River (reflecting the relative expense of these policies compared to dike increases). Of interest to Gorssel are policies O8 and O10 (highlighted yellow in the table). These two policies are favourable for Gorssel and Overijssel (given minimal investment costs in Gorssel) and present an opportunity to start discussions with Overijssel Province to pursue such a policy.

\subsection{Re-evaluation under other formulations}
For the quantitave aspect of synthesis, we considered and compared the top five policies from Deventer and Overijssel in terms of their performance against Gorssel's outcomes of interest. Figure \ref{fig:cost-pol-g} shows the costs to Gorssel for Deventer and Overijssel's preferred policies. As expected, the costs for Deventer's policies were considerably higher than those deemed acceptable by Overijssel. For these reasons, we recommend that Gorssel focuses on engaging with Overijssel first in policy negotiations regarding Gorssel's preferences. These results suggest limited opportunities for coalition-forming with Deventer at the Overijssel-region scale and indicate a need for mediation from Overijssel to find a 'fair' distribution of costs within the region.

\begin{figure}[h!]
    \centering
    \includegraphics[width=\textwidth]{report/figures/results/spreads/cost_policies_Gorssel.png}
    \caption{Costs incurred by Gorssel for Deventer's and Overijssel's top five most robust policies.}
    \label{fig:cost-pol-g}
\end{figure}

Figure \ref{fig:EAD-spread} shows the spread of Gorssel's Expected Annual Damage when Deventer and Overijssel's policies are applied. The vertical axis shows the standard deviation in values for damage across a sweep of uncertainties. Interestingly, Deventer's policies generated less variable outcomes for Gorssel, suggesting that some measure of flood mitigation infrastructure in Gorssel (which is missing from Overijssel's policies) would prevent excess uncertainty in potential damages.

\begin{figure}[h!]
    \centering
    \includegraphics[width=0.6\textwidth]{report/figures/results/spreads/EAD-spread-g.png}
    \caption{Expected Annual Damage incurred by Gorssel for Deventer's and Overijssel's top five most robust policies, values on the vertical axis are in Euros.}
    \label{fig:EAD-spread}
\end{figure}

For this reason, we propose that Gorssel could negotiate with Deventer to advocate that Overijssel consider policies with more flood protection measures, mitigating this uncertainty in damage outcomes.

Additional plots of the effects of Deventer's and Overijssel's policies on Gorssel's outcomes are found in Appendix \ref{a:impacts_policies}.
The results for "Difference in Expected Annual Damage (EAD) between Gorssel and Deventer" and "Difference in Expected Number of Deaths between Gorssel and Deventer" did not show any noteworthy results in this round of the analysis.  
\subsection{Summary of Policy Recommendations}
Based on the above results, there are four key pieces of policy advice for the Municipality of Gorssel:
\begin{itemize}
    \item Modest dike increases in Gorssel and Deventer in the first planning step present a robust solution to minimise flood risk in Gorssel.
    \item Gorssel should prompt the Overijssel regional government to increase the Early Warning System time frame to 3 days, or support already existing efforts to that end.
    \item The current scoping of the problem suggests that Room for the River projects are not favourable due to relative costs (compared to dike increases). However, the sensitivity of proposed policies to uncertainty in the probability of a dike failure suggests that these options may warrant further exploration as a policy option.
    \item We recommend - should Gorssel pursue dike heightening projects and EWS increases - that they complement these policies with other activities to improve their real-world robustness. For example, instantiating improved monitoring and evaluation of dike strength to proactively minimise uncertainty and risk, and/or public relations/community engagement strategies to improve the uptake and acceptance of the increased (and potentially more frequent) warnings.
\end{itemize}