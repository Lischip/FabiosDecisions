\section{Results}
\label{s:results}
%Results; this should be a readable summary of the results from applying the approach. Don’t pursue death by figures but carefully select what visualisations (figures, tables) are functional for telling your story and logically lead to the main conclusions and policy advice?

% Convincing story, consistent with approach using carefully designed visuals and tables to support narrative

\subsection{Uncertainty Analysis}
Uncertainty Analysis revealed uncertainties in these ranges: x-y

The scenarios selected were based on understanding both how the policies perform under these uncertainty ranges, as well as under a diverse set of 'better' conditions. As such, the two 'worst case' scenarios fall within this uncertainty range, while the remaining scenarios fall across other ranges.


\subsection{Robust Decision Making}
%Look at the policies for the five different scenarios, and examine tradeoffs using different robustness metrics.
For the robust decision making process two metrics were used: Satisficing and maximum regret. 
%#################################################################
%   REGRET AND SATISFICING - GORSSEL
%#################################################################
\subsubsection{Gorssel}
\textbf{Satisficing results} \newline
The results from this analysis: The results for Gorssels satisficing analysis with the domain criterion are shown in \autoref{fig:domain_criterion_gorssel}. The graph shows that there are several policies that 
Gorssel only has regret for their total costs, but this regret is negative, so ?
Gorssel has several policies that come above the now set threshold. These three policies results show no damage or loss of life for Gorssel, shown in Table \ref{tab:Gorssel}. Two of them have the same total costs, and one of them is free? Which makes it a tad weird. 

\textbf{results shown in figures: still have to add the tables which show the policy that is satisficing for every threshold. Also I'll put it nicely tomorrow. }
\begin{itemize}
    \item Gorssel has 27 policies of which in the end only 1 is satisficing for every threshold. This policy is : ....
    \item Gorsse has uncertainty about the total costs that they'll have.
    \item The policies generated for Gorssel show that their policies have high treadeoffs between either deatsh and damages vs total costs. This makes it more difficult to find good policy options. 
\end{itemize}

\begin{figure}[H]
  \centering
  \begin{minipage}[b]{0.4\textwidth}
    \includegraphics[width=1.15\textwidth]{report/figures/results/domain_criterion_Gorssel.png}
    \caption{Results for Gorssels domain criterion}
    \label{fig:domain_criterion_gorssel}
  \end{minipage}
  \hfill
  \begin{minipage}[b]{0.4\textwidth}
    \includegraphics[width=1.15\textwidth]{report/figures/results/regret_figure_Gorssel.png}
    \caption{Results for Gorssel's maximum regret}
    \label{fig:regret_gorssel}
  \end{minipage}
\end{figure}




%#################################################################
%   REGRET AND SATISFICING - DEVENTER
%#################################################################
\subsubsection{Deventer}
\textbf{results shown in figures: still have to add the tables which show the policy that is satisficing for every threshold. Also I'll put it nicely tomorrow. }
\begin{itemize}
    \item Deventer has no regret for their total costs, which is logical if they try to get Gorssel to pay for everything.
    \item Deventer has three policies that come above the now set threshold. Shown in Table \ref{tab:deventer}...
    \item Deventer's results for the domain criterion show that there is only one policy solution for the best option that actually performs okay with most of the reference scenarios with a score around 0.6(damages). The other candidate-solutions score lower. 
    \item all candidate-solutions satisfy the thresholds for Deventer in regards to deaths and total budget.
\end{itemize}

\begin{figure}[H]
  \centering
  \begin{minipage}[b]{0.4\textwidth}
    \includegraphics[width=1.15\textwidth]{report/figures/results/domain_criterion_Deventer.png}
    \caption{Results for Deventer's domain criterion}
    \label{fig:domain_criterion_Deventers}
  \end{minipage}
  \hfill
  \begin{minipage}[b]{0.4\textwidth}
    \includegraphics[width=1.15\textwidth]{report/figures/results/regret_figure_Deventer.png}
    \caption{Results for Deventer's maximum regret}
    \label{fig:regret_Deventers}
  \end{minipage}
\end{figure}

%#################################################################
%   REGRET AND SATISFICING - OVERIJSSEL
%#################################################################
\subsubsection{Overijssel}
\textbf{results shown in figures: still have to add the tables which show the policy that is satisficing for every threshold. Also I'll put it nicely tomorrow. }
\begin{itemize}
    \item Overijsel has regret for their Total cost. ? or at least that is what I think the figure means.
    \item Deventer has two policies that come above the now set threshold. Shown in Table \ref{tab:deventer}...
    \item Overijssel's results for the domain criterion show that there is only one policy solution for the best option that actually performs okay with most of the reference scenarios with a score around 0.6 (total costs). The other candidate-solutions score lower. 
    \item all candidate-solutions satisfy the thresholds for Overijssel in regards to deaths.
    \item for the criteria annual damage, the candidate solutions generated a wide spread of domain-criterion score. The results shown in (SHOULD I SHOW THE TABLE IN THE APPENDIX?) show that the top candidate solutions, that fit most reference scenario's for this threshold, exist out of all possible candidate-solutions combined to their respective(base scenario?(worst/best stuff like that)
\end{itemize}
\begin{figure}[H]
  \centering
  \begin{minipage}[b]{0.4\textwidth}
    \includegraphics[width=1.15\textwidth]{report/figures/results/domain_criterion_Overijssel.png}
    \caption{Results for Overijssel's domain criterion}
    \label{fig:domain_criterion_Overijssels}
  \end{minipage}
  \hfill
  \begin{minipage}[b]{0.4\textwidth}
    \includegraphics[width=1.15\textwidth]{report/figures/results/regret_figure_Overijssel.png}
    \caption{Results for Overijssels maximum regret}
    \label{fig:regret_Overijssels}
  \end{minipage}
\end{figure}

\subsection{Sensitivity Analysis}

To see which policies and uncertainties have a large effect on model behaviour, a sensitivity analysis is carried out. The analysis of these can be viewed per actor in the following section. Firstly, the sensitivity analysis was done with feature scoring without the policies to get a grasp of which uncertainty the actors would be most sensitive to. After, policies are added and analysed to assess whether they would be sensitive or not. The generated figures for the sensitivity analysis can be found in Appendix \textbf{XXX}

\subsubsection{Gorssel}

From the sensitivity analysis for damage and deaths for Gorssel it becomes apparent that they are not only influenced by their own dike failure, but also by that of Deventer. This has to do with the fairness criteria specified by Gorssel, where they strive to achieve fairness between all actors in the Overijssel province. As a result, the failures of \textit{both} cities have the highest impact on the deaths and damage. When adding the policies, we observe that the sensitivity to dike failure remains the highest for both the deaths and damage. 

\subsubsection{Deventer}

For Deventer the biggest threat to damages and deaths is the dike failure of just Deventer. This is in line with expectations, because only their own dikes breaking would impact Deventer. And just as with Gorssel, we can observe that when adding the policies, we see that the outcome is still least robust under dike failure. 

\subsubsection{Overijssel}

For the costs of the province of Overijssel it can be observed that they are sensitive to dike increase. The sensitivity is biggest to dike increase in Gorssel, which is in line with reality as this is the most costly procedure in the province, other than Room for the River. However, because none of the policies for Overijssel recommend Room for the River, there is no observed sensitivity. Furthermore, the entire province shows a higher score on deaths and damages across the province as well when it comes to dike failure. 

\subsection{Policy Comparison}
Here, the top five policies from each actor are considered and compared to investigate opportunities for coalition forming, or to identify sources of tension in the policy-making process.