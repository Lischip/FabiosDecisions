\section{Results}
\label{s:results}
%Results; this should be a readable summary of the results from applying the approach. Don’t pursue death by figures but carefully select what visualizations (figures, tables) are functional for telling your story and logically lead to the main conclusions and policy advice?

% Convincing story, consistent with approach using carefully designed visuals and tables to support narrative

\subsection{Scenario Generation and Discovery}
% Generated 50000 scenarios, then ran PRIM to understand results of no policy.

\subsubsection{Gorssel}
\subsubsection{Deventer}
\subsubsection{Overijssel}


\subsection{Scenario Selection}
%We have all the uncertainties - in ranges - for each actor, and for each objective (because PRIM is objective based). Put these all in a table or figure. Then can say these uncertainties are important for this actor, this actor, this actor, these objectives etc. Hopefully, uncertainties from Gelderland will also pop-up, which will tell us about uncertainties in Gelderland.
Following scenario generation and discovery, more information was known about the types of uncertainties that affect the outcomes of interest for all three actors. This was used to inform the scenarios selected, following the process described in Section \ref{s:approach}. The scenario selection approach resulted in six scenarios for each actor:
\begin{itemize}
    \item Best Case
    \item Low (25th Percentile case)
    \item Middle (50th Percentile case)
    \item High (75th Percentile case)
    \item Worst Casualties
    \item Worst Absolute
\end{itemize}
\subsubsection{Gorssel}



\subsubsection{Deventer}
\subsubsection{Overijssel}

\subsection{Robust Decision Making}
%Look at the policies for the five different scenarios, and examine tradeoffs using different robustness metrics.
\subsubsection{Gorssel}
\subsubsection{Deventer}
\subsubsection{Overijssel}

\subsubsection{Sensitivity Analysis}

\subsubsection{Gorssel}
\subsubsection{Deventer}
\subsubsection{Overijssel}


\subsection{Policy Comparison}
Here, the top five policies from each actor are considered and compared to investigate opportunities for coalition forming, or to identify sources of tension in the policy-making process.