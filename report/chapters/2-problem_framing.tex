\section{Problem Framing}
\label{s:prob_frame}
\subsection{Introduction}
Through the years, the river IJssel has been of incredible importance in Dutch history. In times of prosperity, the river was a trading route. In times of war, it made up a strategically defensible border. Situated along this river, between the two bigger cities of Deventer and Zutphen, one can find the smaller town of Gorssel. Located along the banks of the IJssel river, water management and flood prevention have both been interwoven with the town's existence since the first farmers situated themselves there. However, when in 1993 and 1995, the water rose to dangerous levels in the IJssel, the discussion on water management became a national interest. Thus, the Room for the River initiative was born \parencite{rijke_room_2012}. However, the fight against water continues. With global warming added into the mix, the rivers currently face even greater pressure \parencite{takken_wegen_2009} and a flood risk mitigation plan is key for the future of Gorssel and its farmers. Besides the complexity such a plan inherently has, this particular problem requires a careful examination of the actor field, for there are many involved stakeholders. We can fairly assume that all parties want to prevent a life-threatening flood, but beneath that, all involved parties have their own objectives they aim to achieve.  

This report uses innovative modelling and analysis techniques to analyse options for planning flood protection measures along the IJssel river. As explained, flood management in this region is complex due to significant uncertainties (i.e. flood wave shapes, dike failure probabilities, and breach widths) and conflicting priorities of different stakeholders in the region. This report uses exploratory modelling and analysis to support decision-making under the deep uncertainty this complexity causes. We perform this model-based decision making from the perspective of our problem owner, Gorssel, in addition to two key stakeholders of interest: Overijssel Regional Government and the city of Deventer. We do this to reveal necessary trade-offs and the outcomes of conflicting objectives to inform the policy-making process in the Overijssel region.

%\textcolor{red}{1-2 sentences on categorisation of this as a complex problem. what level of uncertainty is it (see intro to Marchau book)? Why do we need to use DMDU techniques?}
% I wasn't completely in agreement with this but that might be my take

% Problem framing; the decision problem can be structured in many ways. 
% Our problem owner is Gorssel


\subsection{Research Question}

%What do you see as the key objectives and constraints?
Protecting the livelihoods of those that reside in Gorssel is Gorssel's primary objective. The population of Gorssel is mainly comprised of farmers, and their livelihood consists of the crops they organically grow. Gorssel wants to minimise flood risk and expected flood damage while also preventing land encroachment to preserve the amount of land available for farming. Both of these are not favourable but are currently in a trade-off with each other. However, there is a limit to how much land Gorssel wants encroached upon - too much, and the farmers are left without anything. Furthermore, it is also essential that any policy measure treats Gorssel equal to the surrounding cities and that other actors also receive fair treatment. The goal of this report is to find a strategy for Gorssel to navigate these difficult waters. % <-haha i get it

% I'm not sure how explicitly to discuss the constraints 

%How are you framing the problem? 
The research question that is in line with this goal is as follows: 
\begin{quote}
    \textit{How can Gorssel (Dike Ring 4) affordably protect its citizens and businesses from flood risk to a similar level as neighbouring urban areas while still preserving and protecting farmland from encroachment?}
\end{quote}

\noindent This question was informed by the mandate for Gorssel provided as part of the EPA1361 Debate (see Section \ref{s:poli_reflect} for more information), in addition to formal documentation of the Room for the River project, conducted by Rijkswaterstaat between 2010 and 2019 \parencite{rijkswaterstaat_tailor_2013,rijke_room_2012}.

To explore suitable policies for achieving these objectives, we devised a problem formulation in the EMA Workbench.
The prioritised objectives for the Gorssel-perspective problem formulation are:
\begin{itemize}
    \item \textit{Minimise Expected Annual Damage for Gorssel}: This objective reflects Gorssel's desire to generally minimise flood risk while also reflecting the effects of damage to farms and other property
    \item \textit{Minimise Expected Number of Deaths for Gorssel:} This objective reflects Gorssel's desire to minimise flood risk generally.
    \item \textit{Minimise Total Costs:} This objective reflects the desire to reduce the cost impact of flood mitigation measures taken across the entire IJssel River system. Total costs for all actors were considered relevant given the requirement of cost-sharing for flood mitigation activities across the region.
    \item \textit{Difference in Expected Annual Damage (EAD) between Gorssel and Deventer (information only - no objective set)}: This function returns results on the difference in Expected Annual Damages between Gorssel and Deventer, but we do not optimise over it. The results for this feature under different policies provide information on the relative 'fairness' in terms of rural-urban burden-sharing in terms of flood risk and as an additional feature for informing satisficing or regret-based robustness metrics.
    \item \textit{Difference in Expected Number of Deaths between Gorssel and Deventer (information only - no objective set)}: rationale for the inclusion of this metric is as for the difference in EAD.
\end{itemize}
These five core objectives are a refinement of all available objectives. We prioritised these due to the computational limitations of the exploratory modelling and analysis approaches (as explained in Section \ref{s:approach}) and due to limitations in stakeholder comprehension of multiple objectives. As we have chosen not to present the analysis in the format of dynamic adaptive policy pathways, we aggregated all objective outcomes to a single planning step. 
%What levers are relevant, and what is being treated as uncertain.
For Gorssel, the relevant levers were:
\begin{itemize}
    \item Dike Heightening: Gorssel can elect for dike heightening in Gorssel and/or Deventer
    \item Early Warning System: Applied across the whole region
    \item Room for the River: Gorssel can elect for Room for the River projects to be implemented in Gorssel or Deventer.
\end{itemize}

\subsection{Actor Analysis}
%Which other actors are we analysing, and why are they natural coalition partners for Gorssel?
We interpreted the primary analysis of preferred options based on Gorssel's mandate, wherein its goals and preferred outcomes are known. But to conduct a complete analysis, it is essential to understand the goals and objectives of other actors involved, as that might reveal opportunities for coalition-forming or potential adversaries. For this analysis, we selected the city of Deventer and the provincial government of Overijssel as other actors for study. This section of the report motivates this choice through an actor analysis approach. 

\bigskip

\noindent To determine which actors were of interest, we assessed the power and interest and the objective of each actor. These two assessments would tell us which actors are of influence on Gorssel's problem and what their interests are in the problem. Starting, we studied which actors could exert power over the discussion. We display these findings in a power-interest grid from Gorssel's point of view. \\

\begin{figure}[h]
    \centering
    \includegraphics[width=0.7\textwidth]{report/figures/PI grid.png}
    \caption{Initial Power Interest Grid}
    \label{fig:pi-grid}
\end{figure}


%It is important to show an awareness of the political arena within which your problem owner is operating. It is also possible to entertain more than one problem formulation.
%Three problem formulations were developed in the EMA Workbench, to reflect the positions and objectives of Gorssel, and the other two actors.

% Explicit framing of problem from multiple relevant perspective (i.e. rival framings)
% Rival framings considered include Overijssel and Deventer, as these were seen as being the most consequential to our group.
% Rival actors
    % Deventer: utilitarian framing, more people in city! Hence city actually > farmers
    % Overijssel: framing itself as facilitator between Gorssel & Deventer, can take more passive role: city farmers battle it out, we implement whatever you figure out.
% a) Actor Analysis
\noindent As can be deduced, the Room for the River Project is a complex multi-actor initiative, and therefore objectives between the stakeholders will not always align. Knowing that Rijkswaterstaat, Deventer and Overijssel are actors with high interest and high power in the issue means that they are the actors for whom Gorssel should be aware. Knowing this, we will study the actors' objectives as it is important to keep in mind potential rival actors and conceptualise possible arguments and framings they can/will use to counter Gorssel's aims. All involved actors and their objectives for the RfR decision arena are listed in Table \ref{t:actortable}.

\begin{table}[h!]
\caption{-}
\begin{tabular}{p{0.2\textwidth}p{0.3\textwidth}p{0.4\textwidth}}
\hline 
Actor & Interest & Objective \\ \hline
Rijkswaterstaat         & Flood risk mitigation, Maintenance of water infrastructure & designing flood protection with widespread support \\ 
Delta Commission        & Long-term Flood Risk Mitigation & implementing effective and robust flood protection projects \\
Transport Company       & Efficient Road Network & ensuring accessibility/connectivity between various cities \\
Env. Protec. Gr.        & Environmental protection & minimising land use change in protected areas due to RfR \\
Gelderland              & Flood risk mitigation, Approval of residents & ensuring equitable time/cost distribution among provinces and municipalities \\
Cortenoever/Doesburg    & Livability for residents & minimising land use change \\
Overijssel              & Flood risk mitigation, Approval of Residents & ensuring equitable time/cost distribution among provinces and municipalities \\
Deventer                & Livability for residents, Aesthetic quality of the city & improving flood security to residents, minimising land use change \\
Gorssel                & Livelihood of resident farmers & protecting farmland, minimising profit loss \\
\end{tabular}
\label{t:actortable}
\end{table}

\bigskip 
% b) Justification of Rival Actors
\noindent Considering the power interest matrix as shown in Figure \ref{fig:pi-grid} as well as relations between actors, Deventer and Overijssel are chosen as rival actors to explore alternate problem formulations. Since the Overijssel provincial government administers both Gorssel and Deventer, there is the greatest chance of having conflicting goals and objectives regarding land-use change, possibly resulting in adverse outcomes for the client. At the same time, these two actors also present the opportunity to find common ground and form coalitions, support one another at the negotiation table with other provinces, like Gelderland, and ensure that Rijkswaterstaat drafts a mutually beneficial proposal. We have converted the actions of other cities (those in Gelderland province) to uncertainties in the model.

\bigskip 

% c) Political Framing
\noindent Opposing stances from rival actors will be accompanied by an alternate framing of the problem. Rival actors can then use this framing to understand their actions. The city of Deventer, a much larger and older municipality of roughly 100 000 inhabitants, will be able to argue from a utilitarian point of view that it should carry greater weight in decision-making. Due to its cultural heritage, it can also argue that land encroachment would be far more invasive than relocating several farms from the community of Gorssel. 

Overijssel, being the next level administrative body above Gorssel and Deventer, will significantly influence which flood mitigation measures Rijkswaterstaat implements in higher-level policy negotiations. Overijssel is required to find some balance of impacts between urban and rural communities as a provincial government. Due to Deventer's larger economic importance, they may use similar framing to legitimise claims for implementing specific measures over others. Regional policy-makers can also obscure discourse using ill-defined neologisms, for example, hybrid concepts such as "nature development" and "spatial quality" \parencite{warner_implementing_2011}. These empty signifiers combine environmental with economic values to appear holistic and fair. Yet, these frames do not address the social and economic concerns of Gorssel's population and fail to address the increasing divide between urban and rural populations. Gorssel may still refine and challenge these views to specifically address these problems.

\smallskip  

The prioritised objectives for the Deventer-perspective problem formulation are:
\begin{itemize}
    \item Minimise Expected Annual Damage for Deventer: This objective reflects Gorssel's desire to generally minimise flood risk while also reflecting effects of damage to farms and other property
    \item Minimise Expected Number of Deaths for Deventer: This objective reflects Deventer's desire to minimise flood risk generally.
    \item Minimise Total Costs: This objective reflects the desire to reduce the cost impact of flood mitigation measures taken across the entire IJssel River system. Total costs for all actors were considered relevant given the requirement of cost-sharing for flood mitigation activities across the region.
\end{itemize}
For Deventer, the policy lever for dike heightening in Deventer is switched off in the model to reflect Deventer's hard constraint around encroachment of sites of cultural and historical significance in the city centre.

The prioritised objectives for the Overijssel-perspective problem formulation are:
\begin{itemize}
    \item Minimise Expected Annual Damage for Gorssel and Deventer: This objective reflects Gorssel's desire to generally minimise flood risk while also reflecting effects of damage to farms and other property
    \item Minimise Expected Number of Deaths for Gorssel and Deventer: This objective reflects Overijssel's desire to minimise flood risk generally for residents and businesses in the region.
    \item Minimise Total Costs: This objective reflects the desire to reduce the cost impact of flood mitigation measures taken across the entire IJssel River system. Total costs for all actors were considered relevant given the requirement of cost-sharing for flood mitigation activities across the region.
\end{itemize}

As is done for Gorssel, we treat the actions of other actors in Gelderland province as uncertainties in both Deventer's and Overijssel's problem formulations.

\textcolor{red}{Should we include something in the description of Gorssel as a client to say that they have a low risk tolerance or something like that to justify choice of RDM over DAPP?}